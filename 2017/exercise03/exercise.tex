\documentclass[a4paper]{article}
\usepackage[german]{babel}
\usepackage[utf8]{inputenc}
\usepackage[T1]{fontenc}
\usepackage{tikz}
\usepackage{pgfplots}
\usepackage{listings}


\begin{document}
\vspace*{-3cm}

\raggedleft
\textbf{Felix Fröhlich} \\
\textbf{Sönke Behrendt}

\centering


~\\~\\
\begin{Large}
\textbf{Exercise 3}
\end{Large}

~\\~\\
\raggedright
\textbf{Features}\\
Lowest Green pixel value:\\
For this feature the value of pixel with the lowest green value is extracted.
~\\ ~\\

Lowest red pixel value:\\
For this feature the value of pixel with the lowest red value is extracted.
~\\ ~\\

Pixels with low red and green values:\\
This feature is generated by counting the number of pixels of an image with red and green values below a certain threshold. The threshold is set to 120.
~\\ ~\\

Dark pixel spots:
The number of dark green pixels surrounded by lighter pixels is a good indicator for the chagas parasite.
~\\ ~\\

Pixels with low green values:
For this feature the number of pixels with a very low green value are counted.

~\\

Model:\\
Covariance matrix:
\begin{lstlisting}
[[  900.89152542   868.0559322   -160.01355932  -119.82711864
-158.14067797]
[  868.0559322   1011.71158192  -148.3259887   -125.91638418
-186.23418079]
[ -160.01355932  -148.3259887    167.23276836     4.15367232    86.5519774 ]
[ -119.82711864  -125.91638418     4.15367232    28.43615819
8.63954802]
[ -158.14067797  -186.23418079    86.5519774      8.63954802
117.91497175]]
\end{lstlisting}

~\\

Positive mean vector:
$[90.733333333333334, 47.166666666666664, 9.4666666666666668, 9.1333333333333329, 10.699999999999999]$
~\\

Negative mean vector:
$[139.86666666666667, 102.8, 2.0666666666666669, 0.73333333333333328, 1.6666666666666667]$


\end{document}

